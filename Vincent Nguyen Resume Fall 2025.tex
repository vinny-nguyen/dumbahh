    \documentclass[10pt, letterpaper]{article}

% Packages:
\usepackage[
    ignoreheadfoot, % set margins without considering header and footer
    top=2 cm, % seperation between body and page edge from the top
    bottom=2 cm, % seperation between body and page edge from the bottom
    left=2 cm, % seperation between body and page edge from the left
    right=2 cm, % seperation between body and page edge from the right
    footskip=1.0 cm, % seperation between body and footer
    % showframe % for debugging 
]{geometry} % for adjusting page geometry
\usepackage{titlesec} % for customizing section titles
\usepackage{tabularx} % for making tables with fixed width columns
\usepackage{array} % tabularx requires this
\usepackage[dvipsnames]{xcolor} % for coloring text
\definecolor{primaryColor}{RGB}{0, 79, 144} % define primary color
\usepackage{enumitem} % for customizing lists
\usepackage{fontawesome5} % for using icons
\usepackage{amsmath} % for math
\usepackage[
    pdftitle={Vincent Nguyen's CV},
    pdfauthor={Vincent Nguyen},
    pdfcreator={LaTeX with RenderCV},
    colorlinks=true,
    urlcolor=primaryColor
]{hyperref} % for links, metadata and bookmarks
\usepackage[pscoord]{eso-pic} % for floating text on the page
\usepackage{calc} % for calculating lengths
\usepackage{bookmark} % for bookmarks
\usepackage{lastpage} % for getting the total number of pages
\usepackage{changepage} % for one column entries (adjustwidth environment)
\usepackage{paracol} % for two and three column entries
\usepackage{ifthen} % for conditional statements
\usepackage{needspace} % for avoiding page brake right after the section title
\usepackage{iftex} % check if engine is pdflatex, xetex or luatex

% Ensure that generate pdf is machine readable/ATS parsable:
\ifPDFTeX
    \input{glyphtounicode}
    \pdfgentounicode=1
    % \usepackage[T1]{fontenc} % this breaks sb2nov
    \usepackage[utf8]{inputenc}
    \usepackage{lmodern}
\fi

% Some settings:
\AtBeginEnvironment{adjustwidth}{\partopsep0pt} % remove space before adjustwidth environment
\pagestyle{empty} % no header or foo    ter
\setcounter{secnumdepth}{0} % no section numbering
\setlength{\parindent}{0pt} % no indentation
\setlength{\topskip}{0pt} % no top skip
\setlength{\columnsep}{0cm} % set column seperation
\makeatletter
\let\ps@customFooterStyle\ps@plain % Copy the plain style to customFooterStyle
\patchcmd{\ps@customFooterStyle}{\thepage}{
    \color{gray}\textit{\small Vincent Nguyen - Page \thepage{} of \pageref*{LastPage}}
}{}{} % replace number by desired string
\makeatother
\pagestyle{customFooterStyle}

\titleformat{\section}{\needspace{4\baselineskip}\bfseries\large}{}{0pt}{}[\vspace{1pt}\titlerule]

\titlespacing{\section}{
    % left space:
    -1pt
}{
    % top space:
    0.3 cm
}{
    % bottom space:
    0.2 cm
} % section title spacing

\renewcommand\labelitemi{$\circ$} % custom bullet points
\newenvironment{highlights}{
    \begin{itemize}[
        topsep=0.10 cm,
        parsep=0.10 cm,
        partopsep=0pt,
        itemsep=0pt,
        leftmargin=0.4 cm + 10pt
    ]
}{
    \end{itemize}
} % new environment for highlights

\newenvironment{highlightsforbulletentries}{
    \begin{itemize}[
        topsep=0.10 cm,
        parsep=0.10 cm,
        partopsep=0pt,
        itemsep=0pt,
        leftmargin=10pt
    ]
}{
    \end{itemize}
} % new environment for highlights for bullet entries


\newenvironment{onecolentry}{
    \begin{adjustwidth}{
        0.2 cm + 0.00001 cm
    }{
        0.2 cm + 0.00001 cm
    }
}{
    \end{adjustwidth}
} % new environment for one column entries

\newenvironment{twocolentry}[2][]{
    \onecolentry
    \def\secondColumn{#2}
    \setcolumnwidth{\fill, 4.5 cm}
    \begin{paracol}{2}
}{
    \switchcolumn \raggedleft \secondColumn
    \end{paracol}
    \endonecolentry
} % new environment for two column entries

\newenvironment{header}{
    \setlength{\topsep}{0pt}\par\kern\topsep\centering\linespread{1.5}
}{
    \par\kern\topsep
} % new environment for the header

\newcommand{\placelastupdatedtext}{% \placetextbox{<horizontal pos>}{<vertical pos>}{<stuff>}
  \AddToShipoutPictureFG*{% Add <stuff> to current page foreground
    \put(
        \LenToUnit{\paperwidth-2 cm-0.2 cm+0.05cm},
        \LenToUnit{\paperheight-1.0 cm}
    ){\vtop{{\null}\makebox[0pt][c]{
        \small\color{gray}\textit{}\hspace{\widthof{}}
    }}}%
  }%
}%

% save the original href command in a new command:
\let\hrefWithoutArrow\href

% new command for external links:
\renewcommand{\href}[2]{\hrefWithoutArrow{#1}{\ifthenelse{\equal{#2}{}}{ }{#2 }\raisebox{.15ex}{\footnotesize \faExternalLink*}}}


\begin{document}
        \newcommand{\AND}{\unskip
        \cleaders\copy\ANDbox\hskip\wd\ANDbox
        \ignorespaces
    }
    \newsavebox\ANDbox
    \sbox\ANDbox{}

    \placelastupdatedtext
    \begin{header}
        \textbf{\fontsize{24 pt}{24 pt}\selectfont Vincent Nguyen}

        \vspace{0.3 cm}

        \normalsize
        \mbox{{\color{black}\footnotesize\faMapMarker*}\hspace*{0.13cm}Waterloo, ON}%
        \kern 0.25 cm%
        \AND%
        \kern 0.25 cm%
        \mbox{\hrefWithoutArrow{mailto:vincent.nguyen3@uwaterloo.ca}{\color{black}{\footnotesize\faEnvelope[regular]}\hspace*{0.13cm}vincent.nguyen3@uwaterloo.ca}}%
        \kern 0.25 cm%
        \AND%
        \kern 0.25 cm%
        \mbox{\hrefWithoutArrow{tel: 5065665675}{\color{black}{\footnotesize\faPhone*}\hspace*{0.13cm}(506) 566-5675}}%
        \kern 0.25 cm%
        \AND%
        \kern 0.25 cm%
        \mbox{\hrefWithoutArrow{https://linkedin.com/in/vincentpnguyen}{\color{black}{\footnotesize\faLinkedinIn}\hspace*{0.13cm}vincentpnguyen}}%
        \kern 0.25 cm%
        \AND%
        \kern 0.25 cm%
        \mbox{\hrefWithoutArrow{https://github.com/vinny-nguyen}{\color{black}{\footnotesize\faGithub}\hspace*{0.13cm}vinny-nguyen}}%
    \end{header}

    \vspace{0.3 cm - 0.3 cm}

    % \section{Welcome to RenderCV!}
        
    %     \begin{onecolentry}
    %         \href{https://rendercv.com}{RenderCV} is a LaTeX-based CV/resume version-control and maintenance app. It allows you to create a high-quality CV or resume as a PDF file from a YAML file, with \textbf{Markdown syntax support} and \textbf{complete control over the LaTeX code}.
    %     \end{onecolentry}

    %     \vspace{0.2 cm}

    %     \begin{onecolentry}
    %         The boilerplate content was inspired by \href{https://github.com/dnl-blkv/mcdowell-cv}{Gayle McDowell}.
    %     \end{onecolentry}

    % \section{Quick Guide}

    % \begin{onecolentry}
    %     \begin{highlightsforbulletentries}


    %     \item Each section title is arbitrary and each section contains a list of entries.

    %     \item There are 7 unique entry types: \textit{BulletEntry}, \textit{TextEntry}, \textit{EducationEntry}, \textit{ExperienceEntry}, \textit{NormalEntry}, \textit{PublicationEntry}, and \textit{OneLineEntry}.

    %     \item Select a section title, pick an entry type, and start writing your section!

    %     \item \href{https://docs.rendercv.com/user_guide/}{Here}, you can find a comprehensive user guide for RenderCV.


    %     \end{highlightsforbulletentries}
    % \end{onecolentry}

    \section{Education}
        
        \begin{twocolentry}{

        \textit{Waterloo, ON}
            
        \textit{Sep 2024 - Present}}
            \textbf{University of Waterloo}

            \textit{Bachelor of Computer Science, Honours, Co-op}
        \end{twocolentry}

        \vspace{0.10 cm}
        \begin{onecolentry}
            \begin{highlights}
                % \item \textbf{Coursework:} Designing Functional Programs, Elementary Algorithm Design and Data Abstraction, Tools and Techniques for Software Development, Linear Algebra 1, Calculus 2, Public Speaking, Digital Lives
                \item \textbf{Extracurriculars:} CS Club Photographer, Data Science Club, Rock Climbing, Muay Thai, Cycling Club
            \end{highlights}
        \end{onecolentry}

    \section{Experience}

        \begin{twocolentry}{
        \textit{Waterloo, ON}    
            
        \textit{Jan 2025 - Present}}
            \textbf{Web Developer}
            
            \textit{Electrium Mobility
}
        \end{twocolentry}

        \vspace{0.10 cm}
        \begin{onecolentry}
            \begin{highlights}
                \item Developing and maintaining the Electrium Shop for the Electrium Mobility design team, streamlining the shopping process of users and designing the user interface using \textbf{React}, \textbf{Tailwind CSS}, and \textbf{Next.js}.
                \item Integrating \textbf{Supabase}'s API to implement product browsing, cart management, connected to a vectorized user authentication database, providing dynamic updates to users and increasing user security by \textbf{80\%}.
            \end{highlights}
        \end{onecolentry}

        \vspace{0.2 cm}

        
        \begin{twocolentry}{
        \textit{Waterloo, ON}    
            
        \textit{Jan 2025 - Present}}
            \textbf{Software Developer}
            
            \textit{University of Waterloo Alternative Fuels Team}
        \end{twocolentry}

        \vspace{0.10 cm}
        \begin{onecolentry}
            \begin{highlights}
                \item Developing a \textbf{Unity}-based Electric Vehicle (EV) driving game with educational content for children in grades 5 - 9 to learn about cars and electric vehicles, using \textbf{C\#}, \textbf{MonoBehaviour}, and \textbf{ScriptableObjects}.
                \item Optimized game dynamics by simulating engine torque, aerodynamic drag, suspension damping, and vehicle acceleration, alongside procedural C\# map generation algorithms, reducing game runtime and lag by \textbf{65\%}.
 
            \end{highlights}
        \end{onecolentry}
        
    \section{Projects}

       \begin{twocolentry}{
         \textit{Waterloo, ON}    
            
        \textit{Feb 2025}}
            \textbf{\textbf{TheRiffler }}\mbox{\hrefWithoutArrow{https://github.com/vinny-nguyen/TheRiffler}{\color{black}{\footnotesize\faGithub}}}
            $|$ \emph{Python, C++, Arduino, PyGuitarPro, PySerial, Onshape, Klipper}
            
            \textit{Hack Canada 2025 Finalist}
        \end{twocolentry}
        
        \vspace{0.10 cm}
        \begin{onecolentry}
            \begin{highlights}
                \item Built an \textbf{Arduino Mega}-based self-playing guitar using \textbf{servomotors} to simultaneously pluck strings and press on frets, with custom \textbf{3D-Printed} actuator components designed using \textbf{Onshape} and \textbf{Klipper}.
                \item Integrated \textbf{PyGuitarPro} to build a \textbf{Python} .gp5 parser that converts guitar tablatures into .JSON files, structured with fret and string numbers, start time, duration, and velocity assigned to each note.
                \item Utilized \textbf{SoundDevice} and \textbf{NumPy} audio arrays to parse .JSON musical data and leveraged \textbf{PySerial} to integrate \textbf{C++} serial command sequences, triggering real-time callibrated Arduino servomotor movements.
            \end{highlights}
        \end{onecolentry}

    \vspace{0.2 cm}

\begin{twocolentry}{
        \textit{Waterloo, ON}    
            
        \textit{Oct 2024 - Dec 2024}}
            \textbf{\textbf{WatClub }}\mbox{\hrefWithoutArrow{https://github.com/Brucewang15/WatClub}{\color{black}{\footnotesize\faGithub}}} $|$ \emph{Python, Selenium, React, Next.js, Docker, Django}
            
            \textit{UW Computer Science Club}
        \end{twocolentry}
        
        \vspace{0.10 cm}
        \begin{onecolentry}
            \begin{highlights}
                \item Developed a rating platform for UW clubs, using \textbf{BeautifulSoup} and \textbf{Selenium} to scrape web data, and designed 30+ \textbf{RESTful} API endpoints for user authentication, comments, and real-time data updates.
                \item Built a custom \textbf{TF-IDF}-based search engine by automating a \textbf{CI/CD} predictive model using \textbf{Docker}, \textbf{Github Actions} and \textbf{Django}, reducing deployment times by \textbf{60\%}, and improving relevancy by \textbf{35\%}.
                
            \end{highlights}
        \end{onecolentry}

        \vspace{0.2 cm}

       \begin{twocolentry}{
         \textit{Waterloo, ON}    
            
        \textit{Sep 2024}}
            \textbf{\textbf{IntroSpectacle }}\mbox{\hrefWithoutArrow{https://github.com/vinny-nguyen/IntroSpectacle}{\color{black}{\footnotesize\faGithub}}}
            $|$ \emph{Python, OpenCV, Mediapipe, MongoDB, Cohere, Whisper}
            
            \textit{Hack the North 2024}
        \end{twocolentry}
        
        \vspace{0.10 cm}
        \begin{onecolentry}
            \begin{highlights}
                \item Developed a real-time facial detection system that helps users remember names, conversations, and details about a person during social interactions by utilizing \textbf{Cohere} and \textbf{Whisper AI} for analyzing transcriptions.
                \item Integrated \textbf{OpenCV}, \textbf{Mediapipe}, and \textbf{PyAudio} for synchronized audio-visual capture and \textbf{MongoDB} for storing facial
recognition data, showing real-time past conversation details to facilitate memory recall.
            \end{highlights}
        \end{onecolentry}
    \section{Skills}

        \begin{onecolentry}
            \textbf{Languages:} Python, C, C++, C\#, Java, HTML5, CSS3, JavaScript, Racket, MATLAB/Simulink
        \end{onecolentry}

        \vspace{0.1 cm}

        \begin{onecolentry}
            \textbf{Technologies:} React, Next.js, Tailwind CSS, Django, PyAudio, PySerial, OpenCV, BeautifulSoup, Selenium \end{onecolentry}

        \vspace{0.1 cm}

        \begin{onecolentry}
            \textbf{Tools:} Git, Docker, Bash, Linux, MongoDB, Supabase, Whisper, Mediapipe, Arduino, Onshape, Klipper\end{onecolentry}

\end{document}
